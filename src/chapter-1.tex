% !TEX root = main.tex

\section{Общие понятия}

Изначально существовал только \textit{натурный или реальный эксперимент}, т.\,е. эксперимент, который проводился над реальными вещами или существами в жизни. Эволюцией данного эксперимента является \textit{вычислительный эксперимент}. Данный эксперимент основывается на моделях. Чтобы произвести вычислительный эксперимент необходим программный комплекс, а в свою очередь для программного комплекса нужны алгоритмы, которые основаны на методах. Методы описывают расчётную схему, которая близка к реальному объекту. Таким образом примитивный вычислительный эксперимент строиться следующим образом:

\[
	\text{объект} \to \text{расчётная схема} \to \text{методы} \to \text{алгоритмы} \to \text{программа, отладка}
\]

Вычислительный эксперимент в конечном счёте существенно экономит время и затраты, в отличие от реального эксперимента.

\section{Классификация моделей}

\begin{enumerate}
	\item материальные;
	\begin{enumerate}
		\item физические;
		\item геометрические;
		\item аналоговые;
	\end{enumerate}
	\item абстрактные;
	\begin{enumerate}
		\item интуитивные;
		\item символьные;
		\item графические;
		\item математические;
		\begin{enumerate}
			\item функциональные;
			\item идентификационные;
			\item имитационные;
		\end{enumerate}
	\end{enumerate}
	\item модели суждения.
\end{enumerate}