% !TEX root = main.tex

\section{Общие понятия}

Изначально существовал только \textit{натурный или реальный эксперимент}, т.\,е. эксперимент, который проводился над реальными вещами или существами в жизни. Эволюцией данного эксперимента является \textit{вычислительный эксперимент}. Данный эксперимент основывается на моделях. Чтобы произвести вычислительный эксперимент необходим программный комплекс, а в свою очередь для программного комплекса нужны алгоритмы, которые основаны на методах. Методы описывают расчётную схему, которая близка к реальному объекту. Таким образом примитивный вычислительный эксперимент строиться следующим образом:

\[
	\text{объект} \to \text{расчётная схема} \to \text{методы} \to \text{алгоритмы} \to \text{программа, отладка}
\]

Вычислительный эксперимент в конечном счёте существенно экономит время и затраты, в отличие от реального эксперимента.

\section{Классификация моделей}

\begin{enumerate}
	\item материальные;
	\begin{enumerate}
		\item физические;
		\item геометрические;
		\item аналоговые;
	\end{enumerate}
	\item абстрактные;
	\begin{enumerate}
		\item интуитивные;
		\item символьные;
		\item графические;
		\item математические;
		\begin{enumerate}
			\item функциональные;
			\item идентификационные;
			\item имитационные;
		\end{enumerate}
	\end{enumerate}
	\item модели суждения.
\end{enumerate}




вторая лекция

третья лекция

\begin{enumerate}
	\item Аппроксимация функций:
	\begin{enumerate}
		\item Интерполяция:
		\begin{enumerate}
			\item Линейная:
			\begin{enumerate}
				\item Полином Ньютона;
				\item Полином Лагранжа;
				\item Полином Эрмита;
				\item Сплайнами;
			\end{enumerate}
			\item Нелинейные;
		\end{enumerate}
		\item Наилучшее среднеквадратичное приближение;
	\end{enumerate}
	\item Численное дифференцирование
	\begin{enumerate}
		\item Дифференцирование полинома;
		\item Сеточные формулы (на основе разложения в ряды Тейлора);
		\item Дифференцирование предварительно сглаженной кривой;
	\end{enumerate}
	\item Численное интегрирование:
	\begin{enumerate}
		\item Формула Ньютона---Котеса;
		\item Метод Гаусса;
		\item Метод Рунге;
	\end{enumerate}
	\item Решение СЛАУ:
	\begin{enumerate}
		\item Прямые методы (класс Гаусса);
		\item Итерационные;
	\end{enumerate}
	\item Обычное дифференциальное уравнение (ОДУ):
	\begin{enumerate}
		\item Задачка Коши;
		\item Краевая задача;
	\end{enumerate}
	\item Уравнение в частных производных (УЧП)
	\begin{enumerate}
		\item Эллиптические $\underbrace{\frac{\delta^2 U}{\delta x_1^2} + \frac{\delta^2 U}{\delta x^2_2} + \frac{\delta^2 U}{\delta x^2_3}}_{\Delta U} = f(x_1\, x_2\, x_3\, U)$;
		\item Параболические $\displaystyle\frac{\delta U}{\delta t} = \Delta U + f(x_1\, x_2\, x_3\, U)$;
		\item Гиперболические $\displaystyle\frac{\delta^2 U}{\delta t^2} = \Delta U + f(x_1\, x_2\, x_3\, U)$;
	\end{enumerate}
	\item Интегро-дифференциальные уравнения.
\end{enumerate}

